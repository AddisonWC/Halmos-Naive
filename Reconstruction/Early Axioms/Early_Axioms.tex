\documentclass{article}
\usepackage{amsmath}
\usepackage{amssymb}
\usepackage{mathtools}
\usepackage[margin=1in]{geometry}

\DeclarePairedDelimiter\prn{(}{)}
\DeclarePairedDelimiter\abs{\lvert}{\rvert}
\DeclarePairedDelimiter\brc{[}{]}
\newcommand{\nl}[0]{\newline}

\begin{document}
\begin{flushleft}

This document is just a bunch of me writing down axioms and shit. \nl

Axiom of extension: Sets are equal if they have the same elements. \nl

Axiom of specification: Within a set, we may make any distinction on the elements of that set, constructing a new set within the old one. However, at this point, the types of distinctions we are able to make are only based on the set-idenity relation = and $\in$, so the sentences we can construct with the "watertight" grammar provided are largely useless. \nl
At this point, we are able to prove that there is no set that could contain all sets: we assert that there is some set $B$ in any given set described by $x \; s.t. \; x \not \in x$. B is a subset of A but it is not a member. If $B \in B$, we find that $B \notin B$ immediately, and likewise if $B \notin B$ we find $B \in B$. Therefore, $B \notin A$, so while we have constructed a subset B, the set A can never have the subset B as an element. \nl
At this point, we can also prove that the existence of any set implies the existence of the null set. \nl

Axiom of pairing: Given any two sets, there is a set that contains both of them. We can reduce this down to the desired ``pair'' set by then specifying $x \in A \; s.t. \; x = a \; or \; x = b$. \nl

Axiom of Unions: Given any set of sets $C$, there is some set that contains all elements $x$ of all sets $A$ that are elements of the parent set $C$. We can reduce this down to the desired actual union by additionally specifying $x \in A \; s.t. \; A \in C$, applying the axiom of specification within the set that the axiom of unions provides us. \nl
We can also construct intersections at this point, again via specification, we take the set B given the axiom of unions and specify $x \; s.t. \; x \in A \kern10px \forall \, A \in C$. \nl
Additionally, we do not actually need our elements to be in the same collection set already-- we can put them in one using the axiom of pairing. \nl




\end{flushleft}
\end{document}