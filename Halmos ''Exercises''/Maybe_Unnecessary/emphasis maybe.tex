\documentclass{article}
\usepackage{amsmath}
\usepackage{amssymb}
\usepackage{mathtools}
\usepackage[margin=1in]{geometry}

\DeclarePairedDelimiter\prn{(}{)}
\DeclarePairedDelimiter\abs{\lvert}{\rvert}
\DeclarePairedDelimiter\brc{[}{]}
\newcommand{\nl}[0]{\newline}

\begin{document}
\begin{flushleft}

\centerline{\textbf{Section 1}}
\fbox{Set inclusion is reflexive} \nl
Any element in A is also in A, since sets are defined by their elements. Thus, $A \subset A$. \nl

\fbox{Set inclusion is transitive}
$A \subset B \kern5px and \kern5px B \subset C$. Any element in A is in B, and any such element is also in C. Thus, any element in A is in C, and $A \subset C$. \nl

\centerline{\textbf{Section 2}}
\fbox{Experimenting with rules of sentence generation} \nl

\centerline{\textbf{Section 3}}

\centerline{\textbf{Section 4}}
\fbox{Associativity of Unions} \nl
WTS \kern20px $x \in \prn*{A \cup B} \cup C \Longrightarrow x \in A \cup \prn*{B \cup C}$: \nl \nl
$x \in \prn*{A \cup B} \cup C \Longrightarrow x \in A \text{ or } x \in \prn*{B \cup C}$ \nl
\hspace*{50px} $x \in C \Longrightarrow x \in \prn*{B \cup C} \Longrightarrow x \in A \cup \prn*{B \cup C}$: \nl
\hspace*{50px} $x \in \prn*{A \cup B} \Longrightarrow x \in A \text{ or } x \in B$: \nl
\hspace*{100px} $x \in A \Longrightarrow x \in A \cup \prn*{B \cup C}$ \nl
\hspace*{100px} $x \in B \Longrightarrow x \in \prn*{B \cup C} \Longrightarrow x \in A \cup \prn*{B \cup C}$ \nl
in all cases, $x \in \prn*{A \cup B} \cup C \Longrightarrow x \in A \cup \prn*{B \cup C}$, thus:
$$\prn*{A \cup B} \cup C \subset A \cup \prn*{B \cup C}$$
Now commute:
$$C \cup \prn*{A \cup B} \subset \prn*{B \cup C} \cup A$$
Now substitute any sets $X, Y, Z$, according to different rules of substitution, into the previous two equations to get:
$$\prn*{X \cup Y} \cup Z \subset X \cup \prn*{Y \cup Z}$$
$$X \cup \prn*{Y \cup Z} \subset \prn*{X \cup Y} \cup Z$$
$$\prn*{X \cup Y} \cup Z = X \cup \prn*{Y \cup Z}$$

\centerline{\textbf{Section 5}}

\centerline{\textbf{Section 6}}

\centerline{\textbf{Section 7}}

\centerline{\textbf{Section 8}}

\centerline{\textbf{Section 9}}

\centerline{\textbf{Section 10}}

\fbox{Functions composition is not commutative} \nl
$f: 2 \rightarrow 2$, set $f(0) = 0, f(1) = 0$. \nl
$g: 2 \rightarrow 2$, set $g(0) = 1, f(1) = 1$. \nl
$f\circ g$ is $2 \rightarrow 2$ s.t. $f \circ g(0) = 1, f \circ g (1) = 1$, but \nl
$g \circ f$ is $2 \rightarrow 2$ s.t. $g \circ f(0) = 0, g \circ f (1) = 0$. \nl 

\end{flushleft}
\end{document}